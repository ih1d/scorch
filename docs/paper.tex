% Author: Isaac H. Lopez Diaz <isaac.lopez@upr.edu>
% Description: Semantics for a ML PL

\documentclass{article}

\title{A Programming Language Paradigm \\ for Machine Learning}

\author{ISAAC H. LOPEZ DIAZ \\
  Department of Computer Science, \\
  Rio Piedras Campus, \\
  Univerisity of Puerto Rico}

\begin{document}
\maketitle
\section{Introduction}
This paper describes the syntax, the operational and denotational semantics, and implementation
of a programming language (PL) paradigm for machine learning (ML). \\

The language is reflective, meaning that the language has a reference to its own semantics. The decision for this is explained, but the motivation
behind such a decision is based on the fact that ML programs tend to show this behaviour. A simple analysis of a TensorFlow or PyTorch program
can demonstrate why a reflective system can be useful, although not necessary. \\

The treatment of such a language lead to the invention of the Lopez-Topology. The reason for a new topology is simply to adjust this
new paradigm of programming to type theory. Although the language discussed is dynamically typed, formal type systems can be
added since the system is a reflective one (here one reason for why a reflective language). \\

Finally, an implementation in Lisp (Scheme) is provided. Lisp was chosen since it has the necessary verbs
to describe such a system.

\section{Syntax}

\section{Semantics}

\subsection{Operational Semantics}

\subsection{Denotational Semantics}

\section{Reflection}

\section{Conclusion}
The system provided is simply a blueprint for extending the mature systems already developed. 

\nocite*{}
\bibliography{refs}
\bibliographystyle{plain}
\end{document}
